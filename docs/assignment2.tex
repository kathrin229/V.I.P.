\documentclass[12pt]{scrartcl}
\title{Assignment 2\\ Interview \& User Data}
\author{Flow Overstack Team}
\date{}
\setlength{\parindent}{0pt}

\usepackage{graphicx}

\begin{document}
\maketitle


\section*{Preparation}
	\subsection*{Information of interest}
		For a successful and handsome application it is essential to know how the users think about the core topics and functions that the application provides. In order to get inspiration about how to structure and design our application and moreover to provide a good user experience satisfying the users' needs, we decided to ask interview questions referring to three different categories:\\
		
		As the V.I.P. app is mainly about presenting historically and scientifically important persons to children, one big topic of questions is ``being famous" and ``famous persons". We want to know what kids think about fame and famous people, which persons are important in their life, who has a roll model function for them and if they could imagine becoming famous one day themselves. This should be a collection of information in general about which kind of persons kids are interested in as well as an ethical discussion about how aware kids are about contradictory famous people of nowadays.\\
		
		Another aspect we are interested in is the kids' smartphone behaviour. Here we especially want to know how well informed children are about smartphone addiction and for which tasks they use their smartphone.\\
		
		Furthermore, we wanted to ask the kids about their attitude towards "learning". Our application should teach and impart knowledge. That is the reason why we are interested in how children prefer to learn, so we can respect these aspects in our design concepts.\\
		
	\subsection*{Types of questions}
		The interview should be comparable between the different groups of children but at the same time structured in a very liberal way. We wanted to ask all kids the same 10 questions. Most of them are open questions, this means we wanted the kids' individual answers and were open for discussions for the topics of each question. In that way we intended to put a guideline into the interview but intended to leave at the same time room for the children's inspirations and suggestions that enrich our font of information according to their needs.


\section*{Interview and Data Analysis}
	\paragraph{Expectations}
		To do
		
	\subsection*{Outcome}
		The kids were very collaborative and forthcoming, allowing us to conduct an impeccable interview and gather a lot of useful data. Some of their answers were very diversified while some others were almost unanimous.
		
		\paragraph{Alan Turing} As an introductory question, we asked all of them if they knew who Alan Turing was: the teacher did, while among the kids only one group answered that ``they had heard the name somewhere but couldn't tell much more" and all the others had never heard of him. 
		
		\paragraph{Role models} This was one of the few questions with the most different results overall. There were famous sportsman like Roger Federer, football players (Messi, Ronaldo), chess players, gymnasts, but also scientists (Einstein, Galileo Galilei, Leonardo Da Vinci, ...) and entrepreneurs like Bill Gates and Steve Jobs. Some kids also mentioned actors, artists, singers and writers like Emma Watson, Beyoncé, Veronica Roth and Van Gogh, or even some youtubers that provide interesting and personal content, not only your ``simple typical Fortnite stream". A girl said that her biggest inspiration and example was her sister.\\
		The general line of thought was that they are inspired by people who did something great for humanity, like scientific discovery or technologic innovations, that leave a sing or a message for the future generations, invest themselves into showing the right example and fight for what they believe in, to achieve their goals.\\
		They also brought up the fact that there are good and bad examples to follow: Chiara Ferragni was mentioned as a good example for how she built her career from nothing, but as a bad personality example; a group said that many of today's rappers were not a good example to follow because ``they only talk about sex, drug and violence".
		
		\paragraph{Being famous} When asked what does it mean to be famous, the answers ranged from ``being a noble inspiration and model for others" to ``having people that know who you are and what you do" to ``yeah, uhm, people with money". The message is that there are various types of famous people: while some are recognised and praised for their actions and achievements, others are just fuzzing around and doing nothing valuable.\\
		According to the kids, being famous has some advantages: you earn a lot of money that can help others and you obtain some kind of exposure or fame that can be used to transmit a message and show the right example. Although it ``would be nice to be recognised", fame also comes with some downside. The examples the kids came up with are stress, bullying, stalkers: it appears that sometimes, being famous is something it can be difficult to deal with.\\
		For all of those reasons, there were mixed responses when asked whether they would like to be famous or not; the most common answer was ``yes, but not too much".
		
		\paragraph{Mobile phone} Of all the 21 kids we interviewed, only one didn't have a smartphone. While this wasn't a big surprise for us, the next questions threw us off quite a lot: not only almost everyone had time restrictions imposed by their parents, but they were more then willing to comply with them, considering them ``right" and being aware that mobile phone addiction is actually a thing. For example, when a kid noticed he was exceeding the time specified by his parents by five or ten minutes, he set up an automated lock of his own will to respect it. Chapeau.
		

\section*{Affinity Diagram Description}
	\begin{itemize}
		\item Orange $\Rightarrow$ categories
		\item Blue-ish color $\Rightarrow$ items
		\item Arrows $\Rightarrow$ connections
	\end{itemize}


\section*{User Requirements}
	Yet to be defined
	
	
\begin{figure}[h]
	\includegraphics[width=\textwidth]{WAAD.jpg}
	\caption{The WAAD}
	\label{WAAD}
\end{figure}

\end{document}