\documentclass[12pt]{scrartcl}
\title{Assignment 5\\ Video Report\\ Paper Prototyping}
\author{\textbf{Flow Overstack Team}\\ Cesana Filippo\\ Folli Gary\\ Hartmann Kathrin\\ Rodolfo Masera Tommaso\\ Stucchi Jacopo\\ Taillefert Stefano}
\date{}
\setlength{\parindent}{0pt}

\usepackage{graphicx}
\usepackage{float}
\usepackage[margin = 3cm]{geometry}


\begin{document}

\maketitle

\tableofcontents

\newpage

\section{Introduction}

	% Describe briefly the app idea and refer to our concept statement

\section{Video Feedback Analysis}
	
	% Analyse and report on data gathered with the one minute video exercise and their implications
	% on our design
	
	\subsection{Expectations versus Reality}
	
		% Summarise what we expected from the children and describe the results we actually got
		% through examples/quotes/etc.
	
	\subsection{Statistics Analysis}
	
		% Spreadsheet of feedback should go in this subsection
		
		\subsubsection*{General tendencies - what’s come out of this globally?}

			Before trying to put our data in a numerical form, we can already look at it to get a general tendency. When we speak about general tendency here, we are not talking about the variance, but simply referring to main ideas or opinions that come out of the data. We were able to identify three general tendencies: 

				\paragraph{Application useless and potentially boring in the middle/long term} 
					That was, unfortunately, one of the main tendencies. A certain part of the opinions has pointed out that our application would be simply useless or boring relatively quickly. The problem was that in all these opinions, almost no real reason was given. One was pointing out that he prefers playing “real games” than educative apps and another more interesting opinion admitted that even though he found the idea was great, he would be bored quite quickly due to the fact that the picture taking process is repetitive.

				\paragraph{Application very interesting and original for the learning process and the discovery of history through it}
					Fortunately, there was another main tendency, even more present than the first one, that finds the application really interesting and the idea original. The kids pointed out that the learning process embedded in the app under the form of an interactive game would not only be interesting in term of knowledge but also nice for comparing the portrait with other people. Globally, the children seem quite interested by the famous people especially for the history behind them; some opinions stated that it would be a funny way to learn history. Two children said that they found the idea great because they would discover new areas of interest. This process of learning and discovery was one of the main objectives of the app and thus, hopefully, the children seem to agree on that. 

				\paragraph{Partial or total misunderstanding of the application concept}
					A third tendency, less present than the other two is a partial or total misunderstanding of the app. In other terms, the children did not understand the idea behind the app. Some of them were honest and wrote it, while others, through their comments, were taking the app for something else (a scanner, a snapchat filter extension with old portraits, …). And in our opinion, this tendency is even more present that we think for the simple reason that a lot of opinions were binaries, that is to say, “yes” or “no”; thus probably, a part of these children did not understand well the concepts and simply gave “no” as an answer. After having discussed between us, it is true that our video was probably kind of unclear for the children that were not in the front of the class and instead of putting a music with texts, a voice would have probably helped them understanding the app better.

		
	\subsection{Suggestion-based Improvement}
	
		% Review the kids' feedback and how we plan to modify the project according to it
	
\section{Paper Prototyping}

	\subsection{Prototype Description}

		% Describe in details the prototype you are creating and how you are going to operate it with 
		% users (peers and children).
		
	\subsection{Key Tasks}
	
		% Show here the list of key tasks (for peers and children) you will use to drive the inspection 
		% process you will run alter on and report in GA6.
			
\section{Conclusion}

	% Describe our expectations for the paper prototype evaluation by the kids/peers

\end{document}